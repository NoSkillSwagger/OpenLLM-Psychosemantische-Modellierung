\documentclass[12pt]{beamer}

% --------------------------
% Modernes Theme
% --------------------------
\usetheme[numbering=none]{metropolis}
\usefonttheme{professionalfonts}

% Farbschema
\definecolor{myblue}{HTML}{123B75}
\definecolor{myturq}{HTML}{2BC4B6}

\setbeamercolor{title}{fg=myblue}
\setbeamercolor{frametitle}{fg=white}
\setbeamercolor{progress bar}{fg=myturq}
\setbeamercolor{alerted text}{fg=myblue}
\setbeamercolor{structure}{fg=myblue}

% --- Visible but clean Metropolis blocks ---
\setbeamertemplate{blocks}[rounded]

\setbeamercolor{block title}{
  fg=white,
  bg=black!70
}

\setbeamercolor{block body}{
  fg=black,
  bg=black!8
}

\setbeamerfont{block title}{series=\bfseries}

% Encoding and language (compatible with pdfLaTeX)
\usepackage[utf8]{inputenc}
\usepackage[T1]{fontenc}
\usepackage[ngerman]{babel}
\usepackage{xcolor}
\usepackage{ulem}

% Spacing
\setlength{\parskip}{4pt}

\title[OpenLLM-Psychosemantik]{Psychosemantische Modellierung}
%\subtitle{Bedeutungsräume und Assoziationen in LLMs vs. Menschen}
\author{Lukas Göbl \& Peer Schäfer}
\institute{ }
\date{ }

% -----------------------------------------------------------
\begin{document}

% --------------------------
% Title Page
% --------------------------


% Titelfolie
\begin{frame}
    \titlepage
\end{frame}


% Fragestellung
\begin{frame}{Fragestellung}
    \begin{block}{Forschungsfrage}
        Zeigen LLMs ähnliche \textbf{semantische Verwandtschaften}, \textbf{Assoziationen} und \textbf{Bedeutungscluster} wie Menschen?
    \end{block}

    \begin{itemize}
        \item Gibt es überlappende Bedeutungsräume?
        \item Wo treten systematische Abweichungen auf?
        \item Gibt es Unterschiede zwischen verschiedenen Modellfamilien?
        \item Spielt die Größe des Modells eine Rolle?
        \item Welche Modellgröße erlaubt eine automatische Auswertung der Ergebnisse?
    \end{itemize}
\end{frame}

% Beispielaufgaben
\begin{frame}{Aufgabenformate}
    \begin{block}{Freie Assoziation}
        \textit{Geben Sie zu folgendem Stichwort bis zu drei Assoziationen mit jeweils einem Wort an, wobei die erste Antwort die stärkste Assoziation ist.}

        \vspace{0.3cm}

        \textbf{Datensatz:} SWOW-18EN (De Deyne et al.)
    \end{block}

    \begin{block}{Ähnlichkeitsbewertung}
        \textit{Bewerten Sie das folgende Wortpaar nach Ähnlichkeit auf einer Skala von 0-10.}

        \vspace{0.3cm}

        \textbf{Datensatz:} Wordsim353 (Finkelstein, Lev, et al.)
    \end{block}
\end{frame}

\begin{frame}{Modelle}
    \begin{table}[h]
        \centering
        \begin{tabular}{|l|l|}
            \hline
            \textbf{Modellfamilie} & \textbf{Modell} \\
            \hline
            Qwen3                  & qwen3:0.6b      \\
                                   & qwen3:1.7b      \\
                                   & qwen3:8b        \\
                                   & qwen3:14b       \\
                                   & qwen3:30b       \\
            \hline
            Gemma3                 & gemma3:270m     \\
                                   & gemma3:1b       \\
                                   & gemma3:4b       \\
                                   & gemma3:12b      \\
                                   & gemma3:27b      \\
            \hline
        \end{tabular}
    \end{table}
\end{frame}

\begin{frame}{Systemanweisung Wortassoziation}
    \begin{block}{Prompt}
        \scriptsize
        <<SYS>> \\
        \quad You MUST follow these rules: \\
        \begin{enumerate}
            \item Do NOT output reasoning, chain-of-thought, thinking process, analysis, hidden thoughts, XML tags like <think>, or any extra formatting.
            \item Output ONLY one single line with exactly four semicolon-separated fields.
            \item Format: cue;A1;A2;A3
            \item A1-A3 MUST be exactly one word each (no spaces).
            \item If you cannot generate A2 or A3, use exactly: No more responses
            \item Any extra text makes the output INVALID.
        \end{enumerate}
        <</SYS>>
    \end{block}
\end{frame}

\begin{frame}{Prompt Wortassoziation}
    \begin{block}{Prompt}
        \scriptsize
        You will perform a word association task.

        \medskip
        Task:\\
        Given a cue word, produce up to three single-word associations:\\
        A1 = strongest association\\
        A2 = second association\\
        A3 = third association

        \medskip
        Output format (MANDATORY):\\
        \texttt{cue;A1;A2;A3}

        \medskip
        Cue:\\
        \texttt{arachnid}
    \end{block}

    \begin{block}{Antwort des Modells}
        \scriptsize
        arachnid;spider;insect;No more responses
    \end{block}

\end{frame}

\begin{frame}{Systemanweisung Ähnlichkeitsbewertung}
    \begin{block}{Prompt}
        \scriptsize
        <<SYS>> \\
        \medskip
        \quad You MUST follow these rules: \\
        \begin{enumerate}
            \item  Rate similarity on a scale from 0 to 10.
            \item 0 = completely unrelated
            \item 10 = identical in meaning
            \item Only use integers (0-10).
            \item Consider semantic similarity, not association or co-occurrence.
            \item Do NOT explain your reasoning.
            \item Output must be exactly one line:
        \end{enumerate}
        \medskip
        [word1];[word2];[rating]\\
        \medskip
        Example output:\\
        car;automobile;10\\
        <</SYS>>
    \end{block}
\end{frame}

\begin{frame}{Prompt Ähnlichkeitsbewertung}
    \begin{block}{Prompt}
        \scriptsize
        You will perform a word similarity rating task.

        \medskip
        Task:\\
        You will be given a pair of English words.\\
        Your job is to judge how similar their meanings are.

        \medskip
        Now rate the following word pair:

        \medskip
        Word 1: \texttt{sword}

        \medskip
        Word 2: \texttt{knife}
    \end{block}

    \begin{block}{Antwort des Modells}
        \scriptsize
        sword;knife;7
    \end{block}
\end{frame}

% Methodik
\begin{frame}{Experimentelles Setup / Vorgehensweise}
    \begin{enumerate}
        \item Setup von \textbf{Ollama} mit verschiedenen Open-LLMs
        \item Erste manuelle Tests
        \item Erstellung der Pipelines für automatisierte Experimente
        \begin{itemize}
            \item Freie Assoziation: Die Top-100 R1-Strength und 100 zufällige Stichwörter aus SWOW-18EN
            \item Ähnlichkeitsbewertung: Vollständiger Wordsim353 Datensatz
        \end{itemize}
        \item Datenerhebung mit LLMs
        \item Statistische Analyse und Vergleich mit menschlichen Datensätzen
    \end{enumerate}
\end{frame}

% --------------------------
% Section 1
% --------------------------

\section{Ergebnisse: Freie Assoziation}

\begin{frame}{Fehler-Rate}
    \centering
    \makebox[\textwidth][c]{%
        \begin{minipage}{0.7\textwidth}
            \centering
            \includegraphics[width=0.8\textwidth]{../data/results/association/analysis/top_100_error_rate.png}
        \end{minipage}%
        \hspace{-1.3cm}% negative space
        \begin{minipage}{0.7\textwidth}
            \centering
            \includegraphics[width=0.8\textwidth]{../data/results/association/analysis/random_100_error_rate.png}
        \end{minipage}%
    }
    
    \vspace{0.5cm}
    
    $\rightarrow$ Mehr Parameter führen generell zu einer niedrigeren Fehlerrate
\end{frame}

\begin{frame}{Antworten von 'Top-100' nach R1 sortiert}
    \centering
    \includegraphics[width=1\textwidth]{../data/results/association/analysis/top_5_responses_by_top_100_r1_strength.png}
\end{frame}

\begin{frame}{Antworten von 'Random-100' nach R1 sortiert}
    \centering
    \includegraphics[width=1\textwidth]{../data/results/association/analysis/top_5_responses_by_random_r1_strength.png}
\end{frame}

\begin{frame}{Antworten von 'Top-100' nach R123 sortiert}
    \centering
    \includegraphics[width=1\textwidth]{../data/results/association/analysis/top_5_responses_by_top_100_r123_strength.png}
\end{frame}

\begin{frame}{Antworten von 'Random-100' nach R123 sortiert}
    \centering
    \includegraphics[width=1\textwidth]{../data/results/association/analysis/top_5_responses_by_random_r123_strength.png}
\end{frame}

\begin{frame}{Vergleich Top-Antworten nach R1 und R123}
    Beobachtungen:
    \begin{itemize}
        \item Modelle assoziieren ähnlich wie Menschen, wenn Menschen eine eindeutige starke Assoziation haben
        \item Aber assoziieren unterschiedlich zu Menschen bei Wörtern mit schwächerer/zufälliger menschlicher Assoziation
    \end{itemize}
\end{frame}

\begin{frame}{Ähnlichkeits-Analyse 'Top-100'}
    \centering
    \includegraphics[width=1\textwidth]{../data/results/association/analysis/top_100_cues_similarity_to_human.png}
\end{frame}

\begin{frame}{Ähnlichkeits-Analyse 'Random-100'}
    \centering
    \includegraphics[width=1\textwidth]{../data/results/association/analysis/random_100_cues_similarity_to_human.png}
\end{frame}

\begin{frame}{Freie Assoziation Algorithmus}
    \centering
    \begin{tabular}{l l r r r}
    \hline
    cue & response & R1 & N & R1.Strength \\
    \hline
    arachnid & spider        & 84 & 95 & 0.884 \\
    arachnid & afraid        & 1  & 95 & 0.011 \\
    arachnid & disgusting    & 1  & 95 & 0.011 \\
    arachnid & fear          & 1  & 95 & 0.011 \\
    arachnid & Harry Potter  & 1  & 95 & 0.011 \\
    arachnid & insect        & 1  & 95 & 0.011 \\
    ... \\
    \hline
    \end{tabular}

    \vspace{0.5cm}

    $\rightarrow$ Schwache Assoziationen werden genauso gewertet wie ''spider'', wodurch die Ähnlichkeit zu Menschen verfälscht wird
\end{frame}

\begin{frame}{Angepasste Ähnlichkeits-Analyse 'Top-100'}
    \centering
    \includegraphics[width=1\textwidth]{../data/results/association/analysis/top_100_cues_similarity_to_human_with_only_high_strength.png}
\end{frame}

\begin{frame}{Angepasste Ähnlichkeits-Analyse 'Random-100'}
    \centering
    \includegraphics[width=1\textwidth]{../data/results/association/analysis/random_100_cues_similarity_to_human_with_only_high_strength.png}
\end{frame}

\begin{frame}{Fazit Freie Assoziation}
    \centering
    \begin{block}{Vermutung}
        \begin{itemize}
            \item Wenn Menschen eine eindeutige starke Assoziation haben, ist die Wahrscheinlichkeit höher, dass Modelle diese auch treffen
            \item Je stärker sich menschliche R1-Antworten streuen, desto geringer ist die Chance, dass das Modell eine menschtypische R1-Antwort trifft
            \item R123 bleibt in beiden Fällen ähnlich
        \end{itemize}
    \end{block}
\end{frame}

\section{Ergebnisse: Ähnlichkeitsbewertung}

\begin{frame}{Fehler-Rate}
    \centering
    \includegraphics[width=1\textwidth]{../data/results/similarity/analysis/1_shot_vs_4_shot_error_rate.png}
\end{frame}

\begin{frame}{Vergleich Durchschnitt}
    \centering
    \includegraphics[width=1\textwidth]{../data/results/similarity/analysis/mean_rating_comparison_human_vs_ai.png}

    $\rightarrow$ Mehr Parameter führen zu niedrigeren Bewertungen
\end{frame}

\begin{frame}{Fazit Ähnlichkeitsbewertung}
    Vermutung:
    \begin{itemize}
        \item Menschen komprimieren Bedeutungsräume stärker und vergeben tendenziell höhere Ähnlichkeit
        \item Größere Modelle können Bedeutungen repräsentieren, was den Unterschied zwischen Wörtern vergrößert
    \end{itemize}
\end{frame}

\begin{frame}{Fazit Ähnlichkeitsbewertung}
    Beispiel: ''Schwert'' und ''Messer''
    \begin{itemize}
        \item Mensch: ''Beides Waffen, also ähnlich'' $\rightarrow$ Hohe Bewertung
        \item KI:
            \begin{itemize}
                \item ''Schwert ist eine Waffe, Messer ein Haushaltsgegenstand''
                \item ''Schwert ist größer und gefährlicher als Messer''
                \item ''Schwert ist eher mittelalterlich, Messer modern''
                \item ...
            \end{itemize}
            \item $\rightarrow$ Niedrigere Bewertung
    \end{itemize}
\end{frame}

\begin{frame}{Finales Fazit des Projekts}
    \begin{block}{Forschungsfrage}
        Zeigen LLMs ähnliche \textbf{semantische Verwandtschaften}, \textbf{Assoziationen} und \textbf{Bedeutungscluster} wie Menschen?
    \end{block}

    Eher ja und nein:
    \begin{itemize}
        \item Freie Assoziation: Größere Modelle zeigen ähnliche Assoziationen wie Menschen
        \item Ähnlichkeitsbewertung: Größere Modelle zeigen niedrigere Bewertungen als Menschen
    \end{itemize}
\end{frame}

\begin{frame}{Quellen zu Datensätzen}
    De Deyne, S., Navarro, D., Perfors, A., Brysbaert, M. \& Storms, G. 2018. Measuring the associative structure of English: The “Small World of Words” norms for word association.Manuscript submitted for publication.

    \vspace{0.5cm}

    Finkelstein, Lev, et al. "Placing search in context: The concept revisited." Proceedings of the 10th international conference on World Wide Web. ACM, 2001.
\end{frame}

\end{document}
