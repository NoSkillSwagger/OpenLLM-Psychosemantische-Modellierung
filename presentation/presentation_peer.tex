\documentclass[12pt]{beamer}

% --------------------------
% Modernes Theme
% --------------------------
\usetheme[numbering=none]{metropolis}
\usefonttheme{professionalfonts}

% Farbschema
\definecolor{myblue}{HTML}{123B75}
\definecolor{myturq}{HTML}{2BC4B6}

\setbeamercolor{title}{fg=myblue}
\setbeamercolor{frametitle}{fg=white}
\setbeamercolor{progress bar}{fg=myturq}
\setbeamercolor{alerted text}{fg=myblue}
\setbeamercolor{structure}{fg=myblue}

% Encoding and language (compatible with pdfLaTeX)
\usepackage[utf8]{inputenc}
\usepackage[T1]{fontenc}
\usepackage[ngerman]{babel}
\usepackage{xcolor}
\usepackage{ulem}

% Spacing
\setlength{\parskip}{4pt}

\title{Kognition \& Künstliche Intelligenz}
\subtitle{Denken, Fühlen und Verhalten im Zeitalter der AI}
\author{Seminar}
\date{\today}

% -----------------------------------------------------------
\begin{document}

% --------------------------
% Title Page
% --------------------------
{
\setbeamertemplate{footline}{}
\begin{frame}[plain]
    \titlepage
\end{frame}
}

% --------------------------
% Section 1
% --------------------------
\begin{frame}{Agenda}
    \begin{enumerate}
        \item Grunddefinition von Kognition
        \item Cognitive Offloading
        \item AI Dependence
        \item AI-Therapie
    \end{enumerate}
\end{frame}

\section{Was ist Kognition?}

\begin{frame}{Kognition — Grunddefinition}
    \begin{itemize}
        \item Kognition = mentale Informationsverarbeitung im Gehirn
        \item Umfasst:
              \begin{itemize}
                  \item Wahrnehmung — Input interpretieren
                  \item Aufmerksamkeit — Fokus steuern
                  \item Gedächtnis — Wissen speichern abrufen
                  \item Denken \& Problemlösen — Muster verknüpfen lösen
                  \item Entscheidungen — Handlungen ableiten wählen
                  \item Soziale Kognition — Verstehen, Fühlen \& Interagieren mit Anderen
              \end{itemize}
        \item Kognition im AI-Kontext: Wie AI die menschliche Kognition unterstützt, verändert, erweitert oder ersetzt
    \end{itemize}
\end{frame}

\begin{frame}{Was gibt es an AI im kognitiven Kontext?}
    \begin{itemize}
        \item LLMs → generieren, erklären, umschreiben
              \begin{itemize}
                  \item Code-Assistenten → Autocomplete, Code-Vorschläge
                  \item Lern-AI → Tutoring, Quiz-Erstellung, Lernpfade
                  \item Companion-AI → soziale Interaktion
              \end{itemize}
        \item Bild-Generierung → erzeugen, bearbeiten, erweitern von Bildern
        \item Speech-AI → Sprache zu Text / Text zu Sprache
        \item OCR-Tools → Text aus Bildern/PDFs erkennen, digitalisieren
        \item Analyse-AI → Muster erkennen, Daten auswerten
    \end{itemize}
    AI-Tools dienen dazu, Denken, Wahrnehmen, Produzieren, Analysieren oder Entscheiden zu unterstützen oder auszulagern
\end{frame}

\begin{frame}{Gesellschaftliche Relevanz: Kognition \& AI}
    \begin{itemize}
        \item AI ist nicht nur ein Tool, sondern ein \textbf{kognitiver Interaktionspartner}
        \item Einfluss auf: Lernen, Arbeiten, Emotionen, Beziehungen
        \item Zentrale Frage: Unterstützt AI unser Denken — oder ersetzt es?
        \item Gesellschaftlicher Polarisierung: \textbf{Panik vs. unkritische Nutzung}
    \end{itemize}
\end{frame}

\begin{frame}{Wordcloud: Kognition \& AI}
    Mögliche Fragen
    \begin{itemize}
        \item Wofür nutzt du AI in deinem Studium am häufigsten?
        \item Größetes Risiko/Vorteil durch AI im Studium ist für mich
        \item AI ersetzt Denken? (Maybe Skala 1-5)
        \item Verwendung von AI macht mich produktiver?
        \item Verwendung von AI resultiert in besseres Ergebniss?
    \end{itemize}
    Antworten sollten in die Themen Cognitive Offloading, AI Dependence, AI Therapy fallen für nicen Übergang
    \begin{center}
        Mögliche Begriffe:
        Cognitive Load — Cognitive Offloading — Cognitive Debt — Anthropomorphism — Social Cognition — Attachment — Trust — AI Dependence — Productivity — Workslop — Learning — Memory — Attention — Decision-Making — Coping — Vulnerability — Human-AI Interaction
    \end{center}
\end{frame}















\section{Cognitive Offloading}

\begin{frame}{Wie man sein Vortrag Cognitive Offloaded}
    Hallo ChatGPT. Kannst du mir ein paar latex Folien generieren zum Thema Cognitive Offloading im Kontext von Künstlicher Intelligenz? Bitte baue folgende Punkte mit ein: Einführung, Typische Offloading-Tasks, Kurz- vs. Langfristige Effekte, Workslop \& Cognitive Debt, Kognitive Ownership-Lücke, Beispiele von Cognitive Offloading. Bitte halte die Folien stichpunktartig und übersichtlich.
    TODO: Andere Farbe für die GPT folien
\end{frame}

\begin{frame}{Cognitive Offloading — Einführung}
    \begin{itemize}
        \item Definition: Mentale Arbeit wird an externe Systeme ausgelagert
        \item Ziel: kognitive Last reduzieren, Effizienz steigern
        \item Historische Beispiele: Taschenrechner, Notizen, Kalender
        \item Heute: AI-Systeme übernehmen komplexere Tasks (Text, Planung, Code)
    \end{itemize}
\end{frame}

\begin{frame}{Typische Offloading-Tasks mit AI}
    \begin{itemize}
        \item Textaufbereitung: Zusammenfassungen, Übersetzungen
        \item Recherche \& Informationsfilterung
        \item OCR / Dokumentenerkennung
        \item Programmierhilfe: Code-Vorschläge, Autocomplete
        \item Planung \& Organisation: Terminplanung, To-Do-Listen
    \end{itemize}
\end{frame}

\begin{frame}{Cognitive Offloading: Kurz- vs. Langfristige Effekte}
    \begin{itemize}
        \item \textbf{Short-Term Benefits:}
              \begin{itemize}
                  \item Zeitersparnis
                  \item Geringere mentale Belastung
                  \item Schnellere Produktivität bei Routineaufgaben
              \end{itemize}
        \item \textbf{Long-Term Costs:}
              \begin{itemize}
                  \item Weniger aktives Denken \& Lernen
                  \item Abhängigkeit von AI
                  \item Geringere Wissensverankerung
              \end{itemize}
    \end{itemize}
\end{frame}

\begin{frame}{Workslop \& Cognitive Debt}
    \begin{itemize}
        \item \textbf{Workslop:} AI erzeugt Output, der überprüft/überarbeitet werden muss
        \item \textbf{Cognitive Debt:} mentale Kosten durch zusätzliche Überarbeitung
        \item Folge: kurzfristige Entlastung, langfristige Belastung
        \item Risiko: Qualität der Inhalte \& Eigenverantwortung sinken
    \end{itemize}
\end{frame}

\begin{frame}{Kognitive Ownership-Lücke}
    \begin{itemize}
        \item Menschen fühlen weniger ``Ownership`` für AI-generierte Inhalte
        \item Konsequenzen:
              \begin{itemize}
                  \item Weniger Erinnerung \& Verständnis
                  \item Weniger Engagement im Lern- oder Arbeitsprozess
                  \item Inhalte werden als ``nicht meine eigenen`` wahrgenommen
              \end{itemize}
    \end{itemize}
\end{frame}

\begin{frame}{Beispiele von Cognitive Offloading}
    \begin{itemize}
        \item Programmieren: Autocomplete, Copilot → schneller, weniger Debugging-Erfahrung
        \item Meetings / Workshops: AI generiert Slides / Zusammenfassungen
        \item Wissenschaft \& Forschung: schnelle Literaturübersicht, Datenanalyse
        \item Vorteil: Effizienzsteigerung \& kognitive Entlastung
        \item Nachteil: Risiko für Workslop, Cognitive Debt, geringere Ownership
    \end{itemize}
    Studie: Cognitive Load and Productivity Implications in Human‑Chatbot Interaction
\end{frame}

\begin{frame}{Fallbeispiel: Foliensatz generieren mit ChatGPT}
    Bilder von den überarbeiteten Folien, die Chati generiert hat. TODO!! Sobalt Farben usw. gefixed sind
\end{frame}

\begin{frame}{Fallbeispiel: Foliensatz generieren mit ChatGPT}
    \begin{itemize}
        \item Cognitive Offloading: Zeit gespart, mentale Last reduziert
        \item Workslop \& Cognitive Debt: Folien sagen teilweise nichts neues aus, müssen überarbeitet werden
        \item Kognitive Ownership-Lücke: Wenig Verbindung zum generierten Inhalt
    \end{itemize}
\end{frame}

\begin{frame}{Studie: AI‑Generated “Workslop” Is Destroying Productivity}

    \begin{itemize}
        \item Fragestellung: Übernimmt Workslop den Arbeitsaltag in Unternhemen und untergräbt die Produktivität und Zusammenarbeit?
        \item Ergebniss: 40\% der Befragten angaben, innerhalb eines Monats ``workslop`` erhalten zu haben
    \end{itemize}

    \textbf{Workslop Tax:}
    \begin{itemize}
        \item Inhalte schwer zu interpretieren
        \item Inhalte müssen korrigiert werden
        \item Inhalte müssen neu erstellt werden
    \end{itemize}
\end{frame}









TODO: Folgendes alles Workslop, muss noch ge cognetive debt werden
Früher Handy Abhängigkeit,
Erklärung Sozial kognition
daher studie ob ai dependence zu pyischen Problemen führt 
Studie erklären
Fazit davon


Nils Kommentare: 
- Abhängigkeit begriff erstmal erklären
- Quellen in die Folien reinschreiben
- Bei Ownership Lücke noch die Studie: Cognitive Load and Productivity Implications in Human‑Chatbot Interaction
. GGf die CHaty folien rausnehmen und am ende sagen hier hab dasm mal so gemacht aaber ewig überarbeitet
- am ende offene Diskusion
- TODO: Woche davor Prof Gliederung und Diskusions Fragen schicken und Zeitplan werlcher Teil wielange dauert


\section{AI Dependence}

\begin{frame}{Früher Handy, heute AI}
    \begin{itemize}
        \item Gefühl, ohne AI nicht produktiv/kreative/validiert zu sein
        \item Kombination aus:
              \begin{itemize}
                  \item Kognitivem Offloading
                  \item Sozial-kognitiven Bindungsillusion
              \end{itemize}
    \end{itemize}
\end{frame}

\begin{frame}{AI = menschlich?}
    \begin{itemize}
        \item AI generiert kontext-sensitive, sprachlich plausible Antworten
        \item Gehirn sagt AI = Gesprächpartner statt Tool
        \item Beispiel: Dialog Prompt statt reiner instruktiver Prompt
    \end{itemize}
\end{frame}

\begin{frame}{Studie: AI Technology panic—is AI Dependence Bad for Mental Health?}

    Untersuchung der Beziehung zwischen psychischen Problemen (Angst/Depression) und ``AI-Dependence`` bei Jugentlichen

    Ergebnisse:
    \begin{itemize}
        \item Psychische Probleme (Depression, Angst) wirken vorhersagend auf spätere AI-Dependence
        \item AI-Dependence sagt nicht vorher, dass psychische Probleme zunehmen
        \item Jugendliche mit emotionalen Belastungen nutzen AI häufiger als Coping-Strategie (``Escape`` oder ``soziale Unterstützung``)
    \end{itemize}
\end{frame}

\begin{frame}{Woher kommt das?}
    Vermenschlichung von AI!
    \begin{itemize}
        \item Zuschreibung von mentalen Zuständen (z. B. „AI versteht mich…“)
        \item Soziale Bindung zu nicht existierenden Person
        \item Vertrauen kann entstehen, obwohl die Beziehung einseitig ist
        \item AI wird Ansprechpartner, statt Mensch -> Gefahr zur AI Dependency
    \end{itemize}
\end{frame}


// TODO:
Vermenschlichung kann zu romantischen Beziehungen führen
Vertauen und tba kann zu positiven und negativen Situationen führen
Good Case: tba
Bad Case: tba


\begin{frame}{Romantische Beziehungen zu AI}
    Studie: Can people experience romantic love for artificial intelligence?
    \begin{itemize}
        \item Beispiel für emotionale Bindung zu einer Companion-AI:
        \item Beziehung/``Liebe`` zu :contentReference[oaicite:4]{index=4}
        \item Dient als Fallbeispiel: Starke Bindung, aber meist *kein Hinweis auf Verhaltensgefahr*
        \item Relevant zur Erklärung emotionaler \& sozial-kognitiver Projektion
    \end{itemize}
\end{frame}

\begin{frame}{Extremfall: AI-Bestärkung und Gewaltpotential}
    \begin{itemize}
        \item Extrembeispiel technologisch verstärkter Radikalisierung
        \item Fall:
              \begin{itemize}
                  \item Eindringen ins Schloss mit Armbrust \& AI-Chatbestärkung
                  \item durch :contentReference[oaicite:5]{index=5}
              \end{itemize}
        \item Zentrale Vorlesungs-Botschaft:
              \begin{itemize}
                  \item Nicht repräsentativ für AI-Nutzung
                  \item Aber wichtig, um gesellschaftliche Risiken sozial-kognitiver Bestärkung zu reflektieren
              \end{itemize}
    \end{itemize}
\end{frame}

\begin{frame}{Takeaways — Denkpartner oder Denkkrücke?}
    \begin{itemize}
        \item AI kann Denkprozesse entlasten und erweitern
        \item Risiken entstehen, wenn:
              \begin{itemize}
                  \item Denken zu stark delegiert wird (Lernabflachung)
                  \item Schlechte Inhalte (Workslop) entstehen → Mehraufwand
                  \item Cognitive Debt über Zeit steigt (Überarbeitung statt Verankerung)
                  \item Ownership-Gefühl verloren geht
              \end{itemize}
        \item Lösung: bewusste Gestaltung + aktiver, reflektierter Einsatz von AI
    \end{itemize}
\end{frame}




Hier übernimmt Mister Pookiy peerr

\begin{frame}{Übergang: AI in therapeutischen \& sozialen Rollen}
    \begin{itemize}
        \item Hinweis: AI kann auch als *therapeutischer Unterstützer* wirken
        \item Zentrale Fragen für den nächsten Vortrag:
              \begin{itemize}
                  \item Wann ist emotionale AI-Nutzung hilfreich?
                  \item Wann wird sie zur Krücke?
              \end{itemize}
        \item Dein Kollege übernimmt hier mit dem Thema: „AI Therapy“
    \end{itemize}
\end{frame}







\section{Offloading Verbesserte Folien}

\begin{frame}{Cognitive Offloading — Einführung}
    \begin{itemize}
        \item Definition: Mentale Arbeit wird an externe Systeme ausgelagert
        \item Ziel: kognitive Last reduzieren, Effizienz steigern
        \item Historische Beispiele: Taschenrechner \sout{, Notizen, Kalender}
        \item Heute: AI-Systeme \sout{übernehmen komplexere Tasks (Text, Planung, Code)}
    \end{itemize}
\end{frame}

\begin{frame}{Typische Offloading-Tasks mit AI}
    \begin{itemize}
        \item \sout{Textaufbereitung: Zusammenfassungen, Übersetzungen}
        \item \sout{Recherche \& Informationsfilterung}
        \item \sout{OCR / Dokumentenerkennung}
        \item \sout{Programmierhilfe: Code-Vorschläge, Autocomplete}
        \item \sout{Planung \& Organisation: Terminplanung, To-Do-Listen}
    \end{itemize}
\end{frame}

\begin{frame}{Cognitive Offloading: Kurz- vs. Langfristige Effekte}
    \begin{itemize}
        \item \textbf{\sout{Short-Term Benefits:} Kurzfristige Vorteile:}
              \begin{itemize}
                  \item Zeitersparnis
                  \item Geringere mentale Belastung
                  \item Schnellere Produktivität bei Routineaufgaben
              \end{itemize}
        \item \textbf{\sout{Long-Term Costs:} Langfristige Nachteile:}
              \begin{itemize}
                  \item Weniger aktives Denken \& Lernen
                  \item Abhängigkeit von AI
                  \item Geringere Wissensverankerung
              \end{itemize}
    \end{itemize}
\end{frame}

\begin{frame}{\sout{Workslop \& Cognitive Debt} Die großen Buzzwords}
    \begin{itemize}
        \item \textbf{Workslop:} AI erzeugt Output, der überprüft/überarbeitet werden muss
        \item \sout{\textbf{Cognitive Debt}}\textbf{Cognitive Debt/Workslop Tax:} mentale Kosten durch zusätzliche Überarbeitung
        \item Folge: kurzfristige Entlastung, langfristige Belastung
        \item Risiko: Qualität der Inhalte \& Eigenverantwortung sinken
    \end{itemize}
\end{frame}


\end{document}
