\documentclass[12pt]{beamer}

% --------------------------
% Modernes Theme
% --------------------------
\usetheme[numbering=none]{metropolis}
\usefonttheme{professionalfonts}

% Farbschema
\definecolor{myblue}{HTML}{123B75}
\definecolor{myturq}{HTML}{2BC4B6}

\setbeamercolor{title}{fg=myblue}
\setbeamercolor{frametitle}{fg=white}
\setbeamercolor{progress bar}{fg=myturq}
\setbeamercolor{alerted text}{fg=myblue}
\setbeamercolor{structure}{fg=myblue}

% --- Visible but clean Metropolis blocks ---
\setbeamertemplate{blocks}[rounded]

\setbeamercolor{block title}{
  fg=white,
  bg=black!70
}

\setbeamercolor{block body}{
  fg=black,
  bg=black!8
}

\setbeamerfont{block title}{series=\bfseries}


% Encoding and language (compatible with pdfLaTeX)
\usepackage[utf8]{inputenc}
\usepackage[T1]{fontenc}
\usepackage[ngerman]{babel}
\usepackage{xcolor}
\usepackage{ulem}
\usepackage{adjustbox}

% Spacing
\setlength{\parskip}{4pt}

% Meta-Daten
\title[OpenLLM-Psychosemantik]{Psychosemantische Modellierung}
%\subtitle{Bedeutungsräume und Assoziationen in LLMs vs. Menschen}
\author{Lukas \and Peer}
\institute{ }
\date{ }

\begin{document}

% Titelfolie
\begin{frame}
    \titlepage
\end{frame}


% Fragestellung
\begin{frame}{Themenstellung}
    \begin{block}{Forschungsfrage}
        Zeigen LLMs ähnliche \textbf{semantische Verwandtschaften}, \textbf{Assoziationen} und \textbf{Bedeutungscluster} wie Menschen?
    \end{block}

    \begin{itemize}
        \item Gibt es überlappende Bedeutungsräume?
        \item Wo treten systematische Abweichungen auf?
        \item Gibt es Unterschiede zwischen verschiedenen Modell Familien?
        \item Spielt die größe des Modells eine Rolle?
        \item Welche Modellgröße erlaubt eine automatische Auswertung der Ergebnisse?
    \end{itemize}
\end{frame}

\begin{frame}{Modelle}
    \begin{table}[h]
        \centering
        \begin{tabular}{|l|l|}
            \hline
            \textbf{Modellfamilie} & \textbf{Modell} \\
            \hline
            Qwen3                  & qwen3:0.6b      \\
                                   & qwen3:1.7b      \\
                                   & qwen3:8b        \\
                                   & qwen3:14b       \\
                                   & qwen3:30b       \\
            \hline
            Gemma3                 & gemma3:270m     \\
                                   & gemma3:1b       \\
                                   & gemma3:4b       \\
                                   & gemma3:12b      \\
                                   & gemma3:27b      \\
            \hline
        \end{tabular}
    \end{table}
\end{frame}

% Beispielaufgaben
\begin{frame}{Aufgabenformate}
    \begin{block}{Freie Assoziation}
        \textit{Geben Sie zu folgendem Stichwort bis zu drei Assoziationen mit jeweils einem Wort an.}

        \vspace{0.3cm}

        \textbf{Datensatz:} Wordsim353 (Finkelstein, Lev, et al.)
    \end{block}

    \begin{block}{Ähnlichkeitsbewertung}
        \textit{Bewerten Sie das folgende Wortpaar nach Ähnlichkeit auf einer Skala von 0-10.}

        \vspace{0.3cm}

        \textbf{Datensatz:} SWOW-18EN (Wulff et al.)
    \end{block}
\end{frame}

\begin{frame}{Prompt -- Systemanweisung}
    \begin{block}{Prompt}
        \scriptsize
        <<SYS>> \\
        \quad You MUST follow these rules: \\
        \begin{enumerate}
            \item Do NOT output reasoning, chain-of-thought, thinking process, analysis, hidden thoughts, XML tags like <think>, or any extra formatting.
            \item Output ONLY one single line with exactly four semicolon-separated fields.
            \item Format: cue;A1;A2;A3
            \item A1-A3 MUST be exactly one word each (no spaces).
            \item If you cannot generate A2 or A3, use exactly: No more responses
            \item Any extra text makes the output INVALID.
        \end{enumerate}
        <</SYS>>
    \end{block}
\end{frame}

\begin{frame}{Prompt -- Wordassoziation}
    \begin{block}{Prompt}
        \scriptsize
        You will perform a word association task.

        \medskip
        Task:\\
        Given a cue word, produce up to three single-word associations:\\
        A1 = strongest association\\
        A2 = second association\\
        A3 = third association

        \medskip
        Output format (MANDATORY):\\
        \texttt{cue;A1;A2;A3}

        \medskip
        Cue:\\
        \texttt{\{cue\}}
    \end{block}

\end{frame}

\begin{frame}{Prompt -- Ähnlichkeitsbewertung}
    \begin{block}{Prompt}
        \scriptsize
        You will perform a word similarity rating task.

        \medskip
        Task:\\
        You will be given a pair of English words.\\
        Your job is to judge how similar their meanings are.

        \medskip
        Now rate the following word pair:

        \medskip
        Word 1: \texttt{\{w1\}}

        \medskip
        Word 2: \texttt{\{w2\}}
    \end{block}


\end{frame}

% Methodik
\begin{frame}{Experimentelles Setup / Verfahrensweise}
    \begin{enumerate}
        \item Setup von \textbf{Ollama} mit verschiedenen Open-LLMs
        \item Erste manuelle Tests
        \item Erstellung der Pipelines für automatiserte Experimente
        \item Datenerhebung mit LLMs
        \item Statistische Analyse und Vergleich mit menschlichen Datensätzen
    \end{enumerate}
\end{frame}

\end{document}
