\documentclass[12pt]{beamer}

% --------------------------
% Modernes Theme
% --------------------------
\usetheme[numbering=none]{metropolis}
\usefonttheme{professionalfonts}

% Farbschema
\definecolor{myblue}{HTML}{123B75}
\definecolor{myturq}{HTML}{2BC4B6}

\setbeamercolor{title}{fg=myblue}
\setbeamercolor{frametitle}{fg=white}
\setbeamercolor{progress bar}{fg=myturq}
\setbeamercolor{alerted text}{fg=myblue}
\setbeamercolor{structure}{fg=myblue}

% --- Visible but clean Metropolis blocks ---
\setbeamertemplate{blocks}[rounded]

\setbeamercolor{block title}{
  fg=white,
  bg=black!70
}

\setbeamercolor{block body}{
  fg=black,
  bg=black!8
}

\setbeamerfont{block title}{series=\bfseries}

% Encoding and language (compatible with pdfLaTeX)
\usepackage[utf8]{inputenc}
\usepackage[T1]{fontenc}
\usepackage[ngerman]{babel}
\usepackage{xcolor}
\usepackage{ulem}

% Spacing
\setlength{\parskip}{4pt}

\title{Kognition \& Künstliche Intelligenz}
\subtitle{Denken, Fühlen und Verhalten im Zeitalter der AI}
\author{Seminar}
\date{\today}

% -----------------------------------------------------------
\begin{document}

% --------------------------
% Title Page
% --------------------------
{
\setbeamertemplate{footline}{}
\begin{frame}[plain]
    \titlepage
\end{frame}
}

% --------------------------
% Section 1
% --------------------------

\section{Ergebnisse: Freie Assoziation}

\begin{frame}{Fehler-Rate}
    \centering
    \makebox[\textwidth][c]{%
        \begin{minipage}{0.7\textwidth}
            \centering
            \includegraphics[width=0.8\textwidth]{../data/results/association/analysis/top_100_error_rate.png}
        \end{minipage}%
        \hspace{-1.3cm}% negative space
        \begin{minipage}{0.7\textwidth}
            \centering
            \includegraphics[width=0.8\textwidth]{../data/results/association/analysis/random_100_error_rate.png}
        \end{minipage}%
    }
    
    \vspace{0.5cm}
    
    $\rightarrow$ Mehr Parameter führt generell zu niedrigere Fehler-Rate
\end{frame}

\begin{frame}{Vergleich Top Antworten nach R1 (KI Top 100)}
    \centering
    \includegraphics[width=1\textwidth]{../data/results/association/analysis/top_5_responses_by_top_100_r1_strength.png}
\end{frame}

\begin{frame}{Vergleich Top Antworten nach R1 (KI Random 100)}
    \centering
    \includegraphics[width=1\textwidth]{../data/results/association/analysis/top_5_responses_by_random_r1_strength.png}
\end{frame}

\begin{frame}{Vergleich Top Antworten nach R1}
    \centering
    \makebox[\textwidth][c]{%
        \begin{minipage}{0.7\textwidth}
            \centering
            \includegraphics[width=0.8\textwidth]{../data/results/association/analysis/top_5_responses_by_top_100_r1_strength.png}
        \end{minipage}%
        \hspace{-1.3cm}% negative space
        \begin{minipage}{0.7\textwidth}
            \centering
            \includegraphics[width=0.8\textwidth]{../data/results/association/analysis/top_5_responses_by_random_r1_strength.png}
        \end{minipage}%
    }
\end{frame}

\begin{frame}{Vergleich Top Antworten nach R123 (KI Top 100)}
    \centering
    \includegraphics[width=1\textwidth]{../data/results/association/analysis/top_5_responses_by_top_100_r123_strength.png}
\end{frame}

\begin{frame}{Vergleich Top Antworten nach R123 (KI Random 100)}
    \centering
    \includegraphics[width=1\textwidth]{../data/results/association/analysis/top_5_responses_by_random_r123_strength.png}
\end{frame}

\begin{frame}{Vergleich Top Antworten nach R123}
    \centering
    \makebox[\textwidth][c]{%
        \begin{minipage}{0.7\textwidth}
            \centering
            \includegraphics[width=0.8\textwidth]{../data/results/association/analysis/top_5_responses_by_top_100_r123_strength.png}
        \end{minipage}%
        \hspace{-1.3cm}% negative space
        \begin{minipage}{0.7\textwidth}
            \centering
            \includegraphics[width=0.8\textwidth]{../data/results/association/analysis/top_5_responses_by_random_r123_strength.png}
        \end{minipage}%
    }
    
\end{frame}

\begin{frame}{Vergleich Top Antworten nach R1 und R123}
    Beobachtungen:
    \begin{itemize}
        \item Modelle asoziert ähnlich zu Menschen gut, wenn sie mit Wörten gepromted werden, bei denen der Mensch eine starke Assoziation hat
        \item Aber asoziert unterschiedlich zu Menschen bei Wörtern mit schwächerer/zufälliger menschlicher Assoziation
    \end{itemize}
\end{frame}

\begin{frame}{Freie Assoziation Top 100}
    \centering
    \includegraphics[width=1\textwidth]{../data/results/association/analysis/top_100_cues_similarity_to_human.png}
\end{frame}

\begin{frame}{Freie Assoziation Random 100}
    \centering
    \includegraphics[width=1\textwidth]{../data/results/association/analysis/random_100_cues_similarity_to_human.png}
\end{frame}

\begin{frame}{Vergleich Freie Assoziation von Top 100 und Random 100}
    Beobachtungen:
    \begin{itemize}
        \item Beide sehr ähnlich zu Menschen
    \end{itemize}

    $\rightarrow$ Bewertungs Algorithmus Fehlerbehaftet!
\end{frame}

\begin{frame}{Freie Assoziation Algorithmus}
    \centering
    \begin{tabular}{l l r r r}
    \hline
    Kategorie & Wort & Wert1 & Wert2 & Anteil \\
    \hline
    arachnid & spider        & 84 & 95 & 0.884 \\
    arachnid & afraid        & 1  & 95 & 0.011 \\
    arachnid & disgusting    & 1  & 95 & 0.011 \\
    arachnid & fear          & 1  & 95 & 0.011 \\
    arachnid & Harry Potter  & 1  & 95 & 0.011 \\
    arachnid & insect        & 1  & 95 & 0.011 \\
    ... \\
    \hline
    \end{tabular}

    \vspace{0.5cm}

    $\rightarrow$ Schwache Assoziationen werden gleich gewertet wie ''spider'', wodurch die Ähnlichkeit zu Menschen verfälscht wird
\end{frame}

\begin{frame}{Angepasste Freie Assoziation Top 100}
    \centering
    \includegraphics[width=1\textwidth]{../data/results/association/analysis/top_100_cues_similarity_to_human_with_only_high_strength.png}
\end{frame}

\begin{frame}{Angepasste Freie Assoziation Random 100}
    \centering
    \includegraphics[width=1\textwidth]{../data/results/association/analysis/random_100_cues_similarity_to_human_with_only_high_strength.png}
\end{frame}

\begin{frame}{Vergleich angepasste Freie Assoziation von Top 100 und Random 100}
    Beobachtungen:
    \begin{itemize}
        \item Top 100 immernoch sehr ähnlich zu Menschen
        \item Random 100 deutlich weniger ähnlich zu Menschen
    \end{itemize}
\end{frame}

\begin{frame}{Fazit Freie Assoziation}
    \centering
    \begin{itemize}
        \item Bei starken Assoziationen zeigen größere Modelle ähnliche Assoziationen wie Menschen
        \item Bei schwächeren/zufälligen Assoziationen weichen Modelle stärker von Menschen ab
    \end{itemize}
\end{frame}

\section{Ergebnisse: Ähnlichkeitsbewertung}

\begin{frame}{Fehler-Rate}
    \centering
    \includegraphics[width=1\textwidth]{../data/results/similarity/analysis/1_shot_vs_4_shot_error_rate.png}
\end{frame}

\begin{frame}{Vergleich Durchschnitt}
    \centering
    \includegraphics[width=1\textwidth]{../data/results/similarity/analysis/mean_rating_comparison_human_vs_ai.png}

    $\rightarrow$ Mehr Parameter führt zu niedrigereren Bewertungen
\end{frame}

\begin{frame}{Fazit Ähnlichkeitsbewertung}
    Vermutung:
    \begin{itemize}
        \item Menschen komprimieren Bedeutungsräume stärker und vergeben tendenziel höhere Ähnlichkeit
        \item Größere Modelle können Bedeutungen repräsentieren, was den Unterschied zwischen Wörtern vergrößert
    \end{itemize}
\end{frame}

\begin{frame}{Fazit Ähnlichkeitsbewertung}
    Beispiel: ''Schwert'' und ''Messer''
    \begin{itemize}
        \item Mensch: ''Beides Waffen, also ähnlich'' $\rightarrow$ Hohe Bewertung
        \item KI:
            \begin{itemize}
                \item ''Schwert ist eine Waffe, Messer ein Haushaltsgegenstand''
                \item ''Schwert ist größer und gefährlicher als Messer''
                \item ''Schwert ist eher mittelalterlich, Messer modern''
                \item ...
            \end{itemize}
            \item $\rightarrow$ Niedrigere Bewertung
    \end{itemize}
\end{frame}

\begin{frame}{Finales Fazit des Projekts}
    \begin{block}{Forschungsfrage}
        Zeigen LLMs ähnliche \textbf{semantische Verwandtschaften}, \textbf{Assoziationen} und \textbf{Bedeutungscluster} wie Menschen?
    \end{block}

    Eher Ja als nein:
    \begin{itemize}
        \item Freie Assoziation: Größere Modelle zeigen ähnliche Assoziationen wie Menschen
        \item Ähnlichkeitsbewertung: Größere Modelle zeigen niedrigere Bewertungen als Menschen
    \end{itemize}
\end{frame}

\end{document}
